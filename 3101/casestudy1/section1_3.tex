\section{Section 3}
\subsection{Kantian Ethics}
Is the company SKI respecting the disabled? Or is SKI taking advantage of the disabled and using their need as a means to make large financial gains? Kant states “Act as to treat humanity, whether in thine own person or in that of any other, in every case as an end withal, never as a means only.” Clearly, SKI is treating the disabled as means. If we try to universalize SKI’s business plan, it becomes impossible and thus breaks down supporting arguments for a monopoly.
\subsection{ACM \& Software Engineering Code of Ethics}
From ACM code of ethics CoE, section 1.1 ``Contribute to society and human well being.'' By restricting access to a technology that would change improve the live of disabled peoples we are not contributing to the well being of humanity. ACM section 1.4 ``Be fair and take action not to discriminate.'' This suggests that the software should have been accessible in the first place, but that is out of the scope of this case study. This point is important because SKI is discriminating when they set the price so high. From Software Engineering CoE 1.07: ``Consider issues of physical disabilities, allocations of resources, economic disadvantage and other factors that can diminish access to the benefits of the software'' is very much applicable to this case. And Finally 1.08 ``Be encouraged to volunteer professional skills to a good cause and contribute to public education concerning the discipline.'' The FROS project is a good cause, and FROS software is not really free, it is made possible by sponsors, donations and volunteers.
\subsection{Other problems \& compromise}
There is the risk that SKI will go out of business; however, careful re-evaluation, re-planning and a change in strategy can prevent this. Plenty of software markets where dozens FROS and cheap alternatives are still dominated by licensed proprietary products. 
\subsection{Prevention}
Companies should expect competition, and they should also be prepared for be under cut. I believe we should not allow patents to get in the way of the development of society. Give credit where credit is due but do not restrict innovation. 