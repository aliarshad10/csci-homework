\section{Section 2}
\subsection{Describe Moral Agents Involved}
Again this case involves three moral agents, the thief, the witness and the victim.
Netscape Communications Corporation, the company who created the proprietary Internet browser Netscape Navigator, is the victim. It is their product being distributed illegally, and their profits the revenues taking the cut. The worker at SciTech Contracting Services is the witness, they have discovered that the thief; Lakeside Industries, is not paying Netscape licensing fees.
\subsection{Identify the Moral Aspects}
The main moral aspect in this situation is stealing. Someone, in this case a Netscape is loosing their property and the money they are entitled to. There exists a side issue of whether or not to get involved. The witness will likely loose their contract and relationship with Lakeside Industries. 
\subsection{State Ethical Implications}
The worker contracted to Lakeside is likely in a position governed by a code of ethics and practising policies. They as a witness should report the crime being committed by Lakeside. However, Netscape is a large international corporation. Lakeside could be one of SciTech's biggest contracts. Reporting Lakeside could in fact ruin the relationship with SciTech Contracting and jeopardize future contracts.  
\subsection{State Choices \& Consequences}
The worker could make one of two choices: report Lakeside (whistle blow) or let it slide. Reporting Lakeside results in an uncertain future; Lakeside could admit it was in error and pay the licensing fees, or they could decide not to and cut off future relations with SciTech. The option of letting it slide is to look the other way. The result is the worker not becoming more involved in the situation, and SciTech-Lakeside relations likely strengthening. The worker however, has already installed software illegally on Lakesides behalf, and this is likely punishable by local or federal law.