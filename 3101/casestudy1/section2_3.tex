\section{Section 3}
\subsection{Applying Kantian Ethics}
Once we categorize property license violations as a form of stealing, we can clearly see Kantian ethics does not look at the end result, but the act itself, and allowing thievery cannot be universalized. Allowing one to steal would leads to not paying for goods and services used or acquired. Not reporting the wrongdoing by Lakeside could eventual result in Lakeside not paying SciTech purely because Lakeside decides what services the wish to pay for.  
\subsection{ACM \& Software Engineering Code of Ethics}
Being a rather textbook example, this case falls under a number of ACM and Software Engineering CoE policies; we shall only list a selection. From the ACM CoE: 1.2 ``avoid harm to others'', 1.3 ``be honest and trustworthy'', 1.5 ``Honour property rights including copyrights and patents''. From the Software Engineering CoE: 1.01 ``Accept full responsibility for ones work'', 2.02 ``Not knowingly use software that is retained illegally or unethically.'' The last point can immediately be applied to this situation. The second last ``Accept full responsibility'' could imply that once our witness has seen the act, he is responsible for knowing his involvement in a crime and reporting it.
\subsection{Other problems \& compromise} 
If this result upsets Lakeside, the relationship between Lakeside and SciTech could be tarnished. We are unaware of the monetary value of the contract, and what percentage of SciTech’s revenue the contract accounts for. This could result in the loss of two businesses. Lakeside after legal fees and penalties are paid, and SciTech who could be putting their largest customer out of business.
\subsection{Prevention}
Companies must realize that stealing software from software companies is equivalent to another company physically stealing from themselves. Our economic and global communities would not be able to exist with thievery being accepted. In recent years, the media has addressed this issue.
