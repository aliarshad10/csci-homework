\section{Read Chapter 10 ``Ethical Interest in Free and Open Source Software'' from \textit{The Handbook of Information and Computer Ethics}}
\begin{center}
``[Free Software] is the morally superior choice to [Open Source Software].''
\end{center}
% State the arguement
% Discuss weather you think the arguement is plausible. 
The argument stated above is given as the conclusion of points presented by Chopra and Dexter. Although I agree that this is plausible, one may disagree. Chopra and Dexter's concluding argument is based on morals and it can be argued to go either way depending on one's moral beliefs. The way that the terms Free Software and Open Source Software are presented in this article, it's analogous to comparing political views, in fact, the political compass could be used to represent software well. As an active contributor to OSS and Free and Open Source Software FOSS, I believe that the attempt to liberate software, by the Free Software Foundation goes too far. Stallman attempted to liberate software yet the Free Softwre Foundation licensing terms are so tight that they seem authoritarian. If we view OSS as Stallman does, ``Open source is a development methodology; free software is a social movement,'' then we can not compare which one is morally superior; one is an apple and the other an orange. The Open Source Initiative disagrees with Stallman on one main point: Stallman believes non-FS to be unethical. Stallman's involvement in creating social change and disregarding the views of others is a move towards the software equivalent of communism. Although communism in theory could work, it has not yet been properly implemented. I do not believe the FSF can be morally superior if it disregards all other views.